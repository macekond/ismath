%% IDEAS:


%% popis řešeného problému (zadání)
%% realizace:
%%    - formalizace
%%    - screenshot jazyka pro zápis pravidel (syntax highlighting)
%%    - realizace nástroje pro ověřování
%% výsledky práce
%%    - návrh matematického přístupu k formulaci pravidel
%%    - rozšiřitelný nástroj

\documentclass{beamer}

\usepackage[utf8]{inputenc}
\usepackage{czech}
\usepackage{graphicx}
\usepackage{amsmath}
\usepackage{amsfonts}
\usepackage{amssymb}
\usepackage{hyperref}

\usetheme{Dresden}
\useinnertheme{rounded}

\newtheorem*{question}{Otázka}
\newtheorem*{response}{Odpověď}
\newtheorem*{motivation}{Motivace}
\newtheorem*{taskspec}{Zadání}

\title{Validace principů objektového návrhu v~kódu}
\author{Martin Vejmelka}

\linespread{1.3}

\begin{document}

\begin{frame}
  \titlepage
\end{frame}

\begin{frame}
  \tableofcontents
\end{frame}

\section{Motivace a zadání práce}
\begin{frame}
\frametitle{Motivace projektu}
    Popsat motivaci a základy řešené problematiky (převzít přímo z práce a zjednodušit na body)
\end{frame}

\begin{frame}
\frametitle{Zadání práce}
\textit{Seznamte se se základními principy používanými při objektově orientovaném návrhu a implementaci, konkrétně s low coupling, high cohesion a Law of Demeter. Popište pravidla, která umožní ověřování těchto principů. Vytvořte nástroj, který umožní analyzovat kód v jazyce Java a vyhodnocovat vámi definovaná pravidla. Činnost nástroje ověřte na vzorových příkladech kódu. Při návrhu nástroje se zaměřte na jeho budoucí rozšiřitelnost.}
\end{frame}

\begin{frame}
\frametitle{Zadání práce - dekompozice}
Zadání práce bylo dekomponováno na následující části:

\begin{itemize}
\item seznámení se s principy objektově orientovaného návrhu (rešeršní část)
\item návrh formalizace popisu pravidel objektově orientovaného návrhu
\item realizace nástroje pro vyhodnocování pravidel popsaných v navrženém formalismu
\item testování výsledného řešení
\end{itemize}

\end{frame}
%% popis řešeného problému (zadání)
%% realizace:
%%    - formalizace
%%    - screenshot jazyka pro zápis pravidel (syntax highlighting)
%%    - realizace nástroje pro ověřování
%% výsledky práce
%%    - návrh matematického přístupu k formulaci pravidel
%%    - rozšiřitelný nástroj

\section{Návrh formalizace pravidel}
\begin{frame}
\frametitle{Návrh formalizace pravidel}
\end{frame}

\section{Realizace vyhodnocovacího nástroje}
\begin{frame}
\frametitle{Realizace vyhodnocovacího nástroje}
\end{frame}

\section{Závěr}
\begin{frame}
\frametitle{Závěr}
\begin{itemize}
\item shrnutí výsledků práce (převzít ze závěru práce)
\end{itemize}
\end{frame}

\section{Otázky}
\begin{frame}
  \frametitle{Otázky}
  \begin{question}
    Jak hodnotíte váš nástroj v porovnání s obdobnými nástroji uvedenými v analytické části práce?
  \end{question}
  \begin{response}
    \begin{itemize}
      \item odpověď here
    \end{itemize}
  \end{response}
\end{frame}

\begin{frame}
  \frametitle{Otázky}
  \begin{question}
    Je možné některý z nástrojů upravit tak, aby byl schopný pracovat s navrženým způsobem ověřování?
  \end{question}
  \begin{response}
    \begin{itemize}
      \item odpověď here
    \end{itemize}
  \end{response}
\end{frame}

\begin{frame}
  \frametitle{Otázky}
  \begin{question}
    Nebylo možné použít některý ze stávajících formalismů (např. Featherweight Java) pro popis pravidel?
  \end{question}
  \begin{response}
    \begin{itemize}
    \item nástroj je realizován na vyšší úrovni abstrakce,
    \item pravidla jsou definována nad obecným grafovým modelem,
    \item FJ by určitě bylo možné využít -- jako elementy v pravidlech by vystupovaly elementy jazyka FJ (příp. FGJ),
    \item pro exaktní formalizaci bychom zvolili vhodné zobrazení programu ve FJ do grafu,
    \item nástroj pracuje nad programy v jazyce Java ($FJ \subset Java$).
    \end{itemize}
  \end{response}
\end{frame}

\section*{Reference}
\begin{frame}
  \frametitle{Reference}
  \bibliographystyle{abbrv} {
    \def\CS{$\cal C\kern-0.1667em\lower.5ex\hbox{$\cal S$}\kern-0.075em $}
    \bibliography{../paper/references}
  }
\end{frame}

\begin{frame}
  \begin{center}
    {\huge Děkuji za pozornost\ldots}
  \end{center}
\end{frame}

\end{document}
