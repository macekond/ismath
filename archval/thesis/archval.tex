%% History:
% Pavel Tvrdik (26.12.2004)
%  + initial version for PhD Report
%
% Daniel Sykora (27.01.2005)
%
% Michal Valenta (3.12.2008)
% rada zmen ve formatovani (diky M. Duškovi, J. Holubovi a J. Žďárkovi)
% sjednoceni zdrojoveho kodu pro anglickou, ceskou, bakalarskou a diplomovou praci

% One-page layout: (proof-)reading on display
%%%% \documentclass[11pt,oneside,a4paper]{book}
% Two-page layout: final printing
\documentclass[11pt,twoside,a4paper]{book}
%=-=-=-=-=-=-=-=-=-=-=-=--=%
% The user of this template may find useful to have an alternative to these
% officially suggested packages:
\usepackage[czech, english]{babel}

\usepackage[T1]{fontenc} % pouzije EC fonty
% pripadne pisete-li cesky, pak lze zkusit take:
% \usepackage[OT1]{fontenc}
\usepackage[utf8]{inputenc}

\usepackage{tikz}
\usepackage{amsmath}
\usepackage{amsthm}
\usepackage{listings}

\newtheorem*{definition}{Definice}
%=-=-=-=-=-=-=-=-=-=-=-=--=%
% In case of problems with PDF fonts, one may try to uncomment this line:
%\usepackage{lmodern}
%=-=-=-=-=-=-=-=-=-=-=-=--=%
%=-=-=-=-=-=-=-=-=-=-=-=--=%
% Depending on your particular TeX distribution and version of conversion tools
% (dvips/dvipdf/ps2pdf), some (advanced | desperate) users may prefer to use
% different settings.
% Please uncomment the following style and use your CSLaTeX (cslatex/pdfcslatex)
% to process your work. Note however, this file is in UTF-8 and a conversion to
% your native encoding may be required. Some settings below depend on babel
% macros and should also be modified. See \selectlanguage \iflanguage.
%\usepackage{czech}  %%%%%\usepackage[T1]{czech} %%%%[IL2] [T1] [OT1]
%=-=-=-=-=-=-=-=-=-=-=-=--=%

%%%%%%%%%%%%%%%%%%%%%%%%%%%%%%%%%%%%%%%
% Styles required in your work follow %
%%%%%%%%%%%%%%%%%%%%%%%%%%%%%%%%%%%%%%%
\usepackage{graphicx}

%%1. odstavec jako v cestine.
%\usepackage{indentfirst}

% thesis formatting macros
\usepackage{k336_thesis_macros}

% work type
\newcommand\TypeOfWork{Diplomová práce} \typeout{Diplomova prace}

% study program
\newcommand\StudProgram{Otevřená informatika, Navazující magisterský}

% study branch
\newcommand\StudBranch{Softwarové inženýrství}

%
% Work title, author, etc.
%
\newcommand\WorkTitle{Validace principů objektového návrhu v kódu}
\newcommand\FirstandFamilyName{Bc. Martin Vejmelka}
\newcommand\Supervisor{Ing. Ondřej Macek}

% Pouzijete-li pdflatex, tak je prijemne, kdyz bude mit vase prace
% funkcni odkazy i v pdf formatu
\usepackage[
  pdftitle={\WorkTitle},
  pdfauthor={\FirstandFamilyName},
  bookmarks=true,
  colorlinks=true,
  breaklinks=true,
  urlcolor=red,
  citecolor=blue,
  linkcolor=blue,
  unicode=true,
]{hyperref}

% Extension posted by Petr Dlouhy in order for better sources reference (\cite{} command) especially in Czech.
% April 2010
% See comment over \thebibliography command for details.

\usepackage[square, numbers]{natbib}             % sazba pouzite literatury
%\usepackage{url}
%\DeclareUrlCommand\url{\def\UrlLeft{<}\def\UrlRight{>}\urlstyle{tt}}  %rm/sf/tt
%\renewcommand{\emph}[1]{\textsl{#1}}    % melo by byt kurziva nebo sklonene,
\let\oldUrl\url
\renewcommand\url[1]{<\texttt{\oldUrl{#1}}>}

\usepackage{nameref}

\begin{document}

\selectlanguage{czech}

\iflanguage{czech}{
  \typeout{************************************************}
  \typeout{Zvoleny jazyk: cestina}
  \typeout{Typ prace: \TypeOfWork}
  \typeout{Studijni program: \StudProgram}
  \typeout{Obor: \StudBranch}
  \typeout{Jmeno: \FirstandFamilyName}
  \typeout{Nazev prace: \WorkTitle}
  \typeout{Vedouci prace: \Supervisor}
  \typeout{***************************************************}
  \newcommand\Department{Katedra počítačové grafiky a interakce}
  \newcommand\Faculty{Fakulta elektrotechnická}
  \newcommand\University{České vysoké učení technické v Praze}
  \newcommand\labelSupervisor{Vedoucí práce}
  \newcommand\labelStudProgram{Studijní program}
  \newcommand\labelStudBranch{Obor}
}{
  \typeout{************************************************}
  \typeout{Language: english}
  \typeout{Type of Work: \TypeOfWork}
  \typeout{Study Program: \StudProgram}
  \typeout{Study Branch: \StudBranch}
  \typeout{Author: \FirstandFamilyName}
  \typeout{Title: \WorkTitle}
  \typeout{Supervisor: \Supervisor}
  \typeout{***************************************************}
  \newcommand\Department{Department of Computer Science and Engineering}
  \newcommand\Faculty{Faculty of Electrical Engineering}
  \newcommand\University{Czech Technical University in Prague}
  \newcommand\labelSupervisor{Supervisor}
  \newcommand\labelStudProgram{Study Programme}
  \newcommand\labelStudBranch{Field of Study}
}

%% % ZADANI (bude vlozeno jako prvni stranka diplomove prace):
%% %
%% % Seznamte se se základními principy používanými při objektově
%% % orientovaném návrhu a implementaci, konkrétně s low coupling, high
%% % cohesion a Law of Demeter. Popište pravidla, která umožní ověřování
%% % těchto principů. Vytvořte nástroj, který umožní analyzovat kód v
%% % jazyce Java a vyhodnocovat vámi definovaná pravidla. Činnost
%% % nástroje ověřte na vzorových příkladech kódu. Při návrhu nástroje se
%% % zaměřte na jeho budoucí rozšiřitelnost.

%
% Title page
%
\coverpagestarts

%
% Acknowledgements
%
\acknowledgements
\noindent
TODO: Zde můžete napsat své poděkování, pokud chcete a máte komu děkovat.

%
% Declaration
%
% TODO: supply valid date
\declaration{V~Praze dne 15.\,5.\,2011}

%
% Abstract
%
\abstractpage

% TODO: Translation of Czech abstract into English.

% Prace v cestine musi krome abstraktu v anglictine obsahovat i
% abstrakt v cestine.
\vglue60mm

\noindent{\Huge \textbf{Abstrakt}}
\vskip 2.75\baselineskip

\noindent
TODO: Abstrakt práce by měl velmi stručně vystihovat její podstatu. Tedy čím se práce zabývá a co je jejím výsledkem/přínosem.

\noindent
Očekávají se cca 1 -- 2 odstavce, maximálně půl stránky.

%
% Table of Contents
%
\tableofcontents

%
% List of Figures
%
\listoffigures

%
% List of Tables
%
\listoftables

%
% Initial settings and start of real work text
%

\mainbodystarts
% horizontalní mezera mezi dvema odstavci
%\parskip=5pt
%11.12.2008 parskip + tolerance
\normalfont
\parskip=0.2\baselineskip plus 0.2\baselineskip minus 0.1\baselineskip

%
% Chapters
%

\chapter{Úvod}

\section{Motivace projektu, záměr práce}
%% motivace projektu (popsat přechod od samotné gramatiky jazyka k dalším pravidlům, která musí výsledný zdrojový kód splňovat)

% popis členění práce - co nalezne čtenář v jednotlivých kapitolách, jak je celé dílo strukturováno

Vymezení práce je graficky znázorněno na obrázku \ref{work_scope}.

TODO: sort and rewrite following paragraphs:

Množina pravidel vymezuje \uv{coding conventions} $\rightarrow$ zpřesnění modelu (omezení množiny instancí daného jazyka) daného gramatikou jazyka.

Analogie:
XML $\rightarrow$ well-formedness vs. validty (validace v tomto kontextu je analogická validaci xml dokumentu)

\begin{figure}[h!]
\centering
\begin{tikzpicture}
\shadedraw  (0,0) rectangle (4,4);
\shadedraw  (4.2,0) rectangle (8.2,4);
\draw  (0,4.2) rectangle (4.0,8.2);
\draw  (4.2,4.2) rectangle (8.2,8.2);

\draw  (2,2) node[text width=2.5cm, text centered] {množina pravidel};
\draw  (6.2,2) node[text width=2.5cm, text centered] {instance splňující pravidla};
\draw  (2,6.2) node[text width=2.5cm, text centered] {gramatika a standardní knihovny};
\draw  (6.2,6.2) node[text width=2.5cm, text centered] {instance jazyka generovaného gramatikou};

\draw (-2,2) node[text width=2.5cm, text centered] {kódové konvence (těžko vynutitelné)};
\draw (-2,6.2) node[text width=2.5cm, text centered] {definice programovacího jazyka};

\draw (2,8.6) node[text width=2.5cm, text centered] {metamodel};
\draw (6.2,8.6) node[text width=2.5cm, text centered] {model};
\end{tikzpicture}
\caption{Grafické znázornění zařazení práce.\label{work_scope}}
\end{figure}

% TODO: integrate following piece of text into main text flow
Množina pravidel může být zadávána neformálně (např.: \uv{třída nesmí záviset na více než čtyřech dalších nesystémových třídách}) nebo pomocí vhodně definovaného formalismu.

\section{Úvod do řešené problematiky}
% lehce rozebrat, čeho se bude práce týkat (spíše hodně highlevel, nezabředávat do detailů)

\chapter{Specifikace cílů práce}

Účelem této sekce je stanovení přesného popisu řešené problematiky a cílů práce. Tyto cíle popíšeme formou požadavků na výsledky práce. Při specifikaci požadavků budeme vycházet ze zadání. Na základě zadání můžeme specifikovat tři oblasti požadavků.

V první oblasti se budeme zabývat požadavky na rešeršní části práce -- analýzu \emph{principů} používaných při \emph{objektově orientovaném návrhu a implementaci}. Část druhá bude obsahovat požadavky na \emph{formalizaci pravidel}, která umožní popsat principy analyzované v části první. Třetí část poskytne rozbor \emph{požadavků na systém}, který by měl demonstrovat vyhodnocování/validaci definovaných pravidel na existujících zdrojových kódech.

Na konci kapitoly rozebereme v rámci jedné sekce existující podobná řešení/nástroje, jejich výhody a nevýhody.

\section{Požadavky na analýzu základních návrhových principů}

TODO: zpracovat
popsat, které návrhové principy budeme analyzovat a z jakého úhlu pohledu

\begin{itemize}
\item rozbor základních principů používaných při objektově orientovaném návrhu a implementaci, konkrétně low coupling, high cohesion a Law of Demeter
\end{itemize}

\section{Požadavky na formalizaci pravidel}

\begin{itemize}
\item vhodně definovat problémovou doménu -- objekty nad nimiž budeme pracovat a základní vztahy mezi nimi
\item definovat jazyk pro zadávání pravidel a specifikace pravidel pro návrhové principy \emph{Law of Demeter}, \emph{low coupling} a \emph{high cohesion}
\end{itemize}

\section{Požadavky na systém pro vyhodnocování pravidel}

TODO: rephrase

Vytvoření nástroje, který umožní, který umožní ověřovat pravidla v existujícím zdrojovém kódu
Globální struktura výsledné práce je na obrázku \ref{requirements-system_structure}.

\begin{figure}[h!]
  \centering
  \includegraphics[width=0.5\textwidth]{./graphs/global_structure.png}
  \caption{Struktura systému.\label{requirements-system_structure}}
\end{figure}

\subsection{Funkční požadavky na výsledný systém}
\begin{itemize}
\item průběžná kontrola projektu a výpis zjištěných porušení pravidel (validace) do výstupního souboru
\item provedení kontroly \uv{on demand} -- uživatel stiskne tlačítko, provede se kontrola projektu proti pravidlům a výstup se zobrazí ve výstupním souboru (např. konzoli)
\end{itemize}

\subsection{Nefunkční požadavky na výsledný systém}
\begin{itemize}
\item systém bude fungovat nad instancemi programovacího jazyka Java 6 (projekty napsané v jazyku Java)
\item systém bude umožňovat definovat další pravidla (požadavek rozšiřitelnosti ze zadání)
\end{itemize}

\section{Rešerše existujících řešení}
\label{requirements-existing_tools}

TODO: možná přidat i rešerši exisujících formalizací pro návrh software (Z-notation, UPPAAL, etc.)

TODO: provest resersi o tom, co vsechny tyto nastroje umi, jejich pozitiva a negativa

TODO: v rychlosti zopakovat rešerši na nové nástroje a konkrétní vlastnosti, které jsou důležité pro tuto práci

\subsection{JDepend}

\begin{itemize}
\item \href{http://www.clarkware.com/software/JDepend.html}{http://www.clarkware.com/software/JDepend.html}
\item nástroj pro testování kvality návrhu
\item pracuje nad \verb+*.class+ soubory (získává data z bytekódu)
\end{itemize}

\subsection{QJ-Pro}
\begin{itemize}
\item \href{http://qjpro.sourceforge.net}{http://qjpro.sourceforge.net}
\end{itemize}

\subsection{DP-Miner}
\begin{itemize}
\item \href{http://www.utdallas.edu/~yxz045100/DesignPattern/DP\_Miner/}{http://www.utdallas.edu/~yxz045100/DesignPattern/DP\_Miner/}
\item hledání návrhových vzorů v existujících projektech
\item článek: Jing Dong and Yajing Zhao, Experiments on Design Pattern Discovery \\ (\href{http://www.utdallas.edu/~jdong/papers/PROMISE07.pdf}{http://www.utdallas.edu/~jdong/papers/PROMISE07.pdf})
\end{itemize}

\subsection{Macker}
\begin{itemize}
\item \href{http://innig.net/macker/}{http://innig.net/macker/}
\item build-time architectural rule checking utility for Java developers
\item zpracovává \verb+*.class+ soubory (bytekód)
\end{itemize}

TODO: complete research on following tools:

\subsection{Squale}
\begin{itemize}
\item \href{http://www.squale.org/}{http://www.squale.org/}
\end{itemize}

\subsection{FindBugs}
\begin{itemize}
\item \href{http://findbugs.sourceforge.net/}{http://findbugs.sourceforge.net/}
\end{itemize}

\subsection{CheckStyle}
\begin{itemize}
\item \href{http://checkstyle.sourceforge.net/}{http://checkstyle.sourceforge.net/}
\end{itemize}

\subsection{PMD}
\begin{itemize}
\item \href{http://pmd.sourceforge.net/}{http://pmd.sourceforge.net/}
\end{itemize}

\subsection{Soot}
\begin{itemize}
\item \href{http://www.sable.mcgill.ca/soot/}{http://www.sable.mcgill.ca/soot/}
\item Soot: a Java Optimization Framework
\end{itemize}

\chapter{Analýza}
%% TODO: zde napsat prehledovy text o tom, co vsechno tato kapitola obsahuje

%% Analýza a návrh implementace (včetně diskuse různých alternativ a volby implementačního prostředí).

%% V části analýzy provést rozbor jednotlivých principů + výsledky rešerší - využívat hojně informace z článků a důsledně citovat zdroje

\section{Analýza principů objektového návrhu}

%% PRINCIPY versus PATTERNY -> citovat clanek

\subsection{Analyzované principy}
Ukázkové návrhové principy analyzované v rámci této práce jsou znázorněny na obrázku \ref{analyzed_principles}. Poznamenejme, že tzv. Demeterův zákon je speciálním případem pravidla pro \uv{low coupling}.

\begin{figure}[h!]
  \centering
  \includegraphics[width=0.5\textwidth]{./graphs/oop_design_principles.png}
  \caption{Znázornění analyzovaných návrhových principů.\label{analyzed_principles}}
\end{figure}

% TODO: v rámci každého návrhového principu uvést příklad porušení
% tohoto principu (případně i příklad, který tento princip dodržuje)

% Příklad pravidla:
% ``Třídy z balíčku A nesmí záviset na jiných konkrétních třídách, ale nejvýše na rozhraních balícků B.''
%  (programování proti rozhraní namísto proti kokrétní implementaci)"

\subsubsection{Low coupling/dependency (nízká závislost/vazba)}
Důležitou návrhovou zásadou je snaha snížit provázanost modulů na minimum. Je možné kategorizovat způsob provázanosti modulů do různých skupin. Následující přehled je převzat z \cite{wiki:coupling} (ještě jemnější dělení je uváděno v \cite{STVR:STVR162}).

\begin{itemize}
\item\emph{Content coupling (nejvyšší forma závislosti)} -- závislost na obsahu modulu -- jeden modul modifikuje nebo se spoléhá na vnitřní fungování jiného modulu (např. přístup k lokálním datům jiného modulu). V důsledku platí, že změní-li se způsob, kterým tento druhý modul produkuje data (umístění, typ, časování), povede to zcela jiste ke změnám v závislém modulu.
\item\emph{Common coupling} -- dva moduly sdílí stejná globální data (např. globální proměnnou), změna sdíleného globálního zdroje implikuje změny všech modulů, které je používají.
\item\emph{External coupling} -- dva moduly sdílí externě definovaný (standardizovaný) datový formát, komunikační protokol nebo rozraní zařízení.
\item\emph{Control coupling} -- jeden modul kontroluje tok druhého tím, že mu posílá informaci o tom, co má konat (např. předání \uv{to-do} příznaku).
  % TODO: CLARIFY
\item\emph{Stamp coupling (Data-structured coupling)} -- mdouly sdílí složenou datovou strukturu a používají pouze její (často odlišnou) část (např. předávání kompletního záznamu funkci, která z něj potřebuje pouze jedno pole). Tato vazba může vést ke změně způsobu, kterým modul čte záznam, protože pole, které tento modul nepotřebuje bylo modifikováno.
  % Stamp coupling is when modules share a composite data structure and use only a part of it, possibly a different part (e.g., passing a whole record to a function that only needs one field of it). This may lead to changing the way a module reads a record because a field that the module doesn't need has been modified.
\item\emph{Data coupling} -- moduly sdílí data pomocí parametrů. Každý parametr je elementární datový typ a jedná se o jediná data, která jsou sdílená (např. předávání celočíselné hodnoty funkci, která spočítá jeho druhou mocninu).
\item\emph{Message coupling (nejnižší forma závislosti)} -- provázanost modulů pouze pomocí zpráv, jedná se o nejnižší úroveň závislosti. Lze jí dosáhnout pomocí decentralizace stavu (u objektů), kde je komunikace dosahováno pomocí parametrů nebo předávání zpráv (\uv{message passing}).
\item\emph{No coupling} -- žádná závislost -- moduly spolu vůbec nekomunikují.
\end{itemize}

%% TODO: sort
%% ## Typy závislostí mezi třídami

%% * třída A dědí ze třídy B
%% * třída A provádí instanciaci třídy B
%% * třída A používá existující intanci třídy B (pracuje s referencí na tuto třídu)

%% Na základě těchto závislostí lze sestavit orientovaný graf. Hrany budeme dále klasifikovat podle toho, o jakou závislost se jedná (dědičnost vs. vyvolání metody).

\subsubsection{High cohesion}

\subsubsection{Law of Demeter}

Existuje několik forem Demeterova zákona \cite{35588}, které jsou vhodné pro různé oblasti aplikace. Tyto typy jsou znázorněny na obrázku \ref{demeter_law_types}.

\begin{figure}[h!]
  \centering
  \includegraphics[width=0.7\textwidth]{./graphs/demeter_law_types.png}
  \caption{Formy Demeterova zákona.\label{demeter_law_types}}
\end{figure}

Pro statickou analýzu lze použít \uv{class} formu Demeterova zákona.

\subsection{Ukázky kódu porušujících některá z pravidel}

\subsubsection{Porušení principu law of Demeter}

%% TODO: use some highlighter
%% TODO: add example of demeter law violation from paper notes
\begin{verbatim}
package handlers;

public class ...

\end{verbatim}

\section{Analýza problematiky v jazyce Java}

\subsection{Statický model programu v Javě}
TODO: pojednání o tom, co všechno může být modelem programu (ast, graf, FSM, programovací jazyky, atd)

TODO: možná přidat poznámku o model driven engineering a jak může být aplikováno právě zde

\subsubsection{Struktura softwarového projektu v Javě}

% TODO: sort and rephrase
\begin{itemize}
\item\verb+*.java+ soubory - v gramatice programovacího jazyka Java 1.5 představují top-level element CompilationUnit \emph{\{(TODO: binární součásti projektu? class files?)\}}
\item další soubory - resources, documentation, \ldots
\end{itemize}

Pro naše potřeby jsou důležité v podstatě pouze kompilační jednotky (java soubory) projektu.

Statický pohled na program - neuvažujeme běh programu. Pracujeme nad definicemi tříd, nikoliv nad jejich instancemi v paměti JVM.

% TODO: aktualizovat -> v konecnem dusledku budeme pracovat i nad projekty v jazyku 1.6 (protoze nam to rozhrani umoznuje)
Budeme pracovat nad gramatikou jazyka Java 1.5. Java verze 6 se liší pouze úpravou standardních API poskytovaných platformou Java. Jazyk jako takový zůstává stejný.

\subsubsection{Syntaktické elementy programovacího jazyka Java}
Grafické znázornění základních syntaktických elementů, jejichž struktura a názvy jsou převzaty z \cite{Gosling:2005:JLS:1036643}, je na obrázku \ref{toplevel_elements}.
% TODO: write some better accompanying text
\begin{figure}[h!]
  \centering
  \includegraphics[width=\textwidth]{./graphs/java_top_elements.png}
  \caption{Struktura základních syntaktických elementů programovacího jazyka Java.\label{toplevel_elements}}
\end{figure}

Pro analýzu založenou na vyhledávání závislostí mezi třídami pro nás bude nejdůležitější syntaktický element \emph{TypeDeclaration}. Tento neterminální symbol se dále přepisuje na symboly uvedené na obrázku \ref{type_declaration_options}.

\begin{figure}[h!]
  \centering
  \includegraphics[width=\textwidth]{./graphs/toplevel_types.png}
  \caption{Rozklad elementu TypeDeclaration.\label{type_declaration_options}}
\end{figure}

%% TODO: zatridit tyto syntakticke elementy, ke kazdemu napsat, jak se bude zpracovavat
%% interfaces
%% enums
%% annotations

%% Modifikátory přístupu (private, public, protected, package private) nám umožní "ořezat" graf závislosti tříd. Je ale možné, že pro některé druhy analýzy bude toto nežádoucí.

%% ## Speciální případy:

%% * statické třídy
%% * statické metody
%% * vnitřní třídy

\subsection{Existující nástroje pro zpracování zdrojových kódů v jazyku Java}

TODO: write some leading text

Možnosti zpracovávání kódu:

TODO: roztřídit a vyházet nesmysly (e.g. JABA, apod.)

TODO: použít jinou vhodnější strukturu místo itemize

\begin{itemize}
\item vlastní hand-written lexikální a syntaktický analyzátor (zbytečně náročné)
\item parser vygenerovaný pomocí některého z dostupných compiler-compiler systémů
  \begin{itemize}
  \item tyto systémy na základě vstupní gramatiky vygenerují frontend pro překladač
  \item lze specifikovat různé akce, které jsou navázány na události vyvolané v průběhu syntaktické analýzy
  \item \emph{JavaCC}
    \begin{itemize}
    \item \href{http://www.cs.purdue.edu/homes/hosking/352/javaccdocs/docindex.html}{http://www.cs.purdue.edu/homes/hosking/352/javaccdocs/docindex.html}
    \end{itemize}
  \item \emph{JastAdd}
    \begin{itemize}
    \item \href{http://jastadd.org/}{http://jastadd.org/}
    \end{itemize}
  \item \emph{JLex}
    \begin{itemize}
    \item A Lexical Analyzer Generator for Java(TM)
    \item \href{http://www.cs.princeton.edu/~appel/modern/java/JLex/}{http://www.cs.princeton.edu/~appel/modern/java/JLex/}
    \end{itemize}
  \item JFlex
    \begin{itemize}
    \item The Fast Scanner Generator for Java
    \item \href{http://jflex.de/}{http://jflex.de/}
    \end{itemize}
  \item JABA
    \begin{itemize}
    \item \href{http://pleuma.cc.gatech.edu/aristotle/Tools/jaba.html}{http://pleuma.cc.gatech.edu/aristotle/Tools/jaba.html}
    \item Java Architecture for Bytecode Analysis
    \item nepoužitelné - dostupná pouze binární verze pro Solaris
    \item pravděpodobně se dále nevyvíjí
    \end{itemize}
  \end{itemize}
\item použití vhodné knihovny
  \begin{itemize}
  \item JavaParser
    \begin{itemize}
    \item \href{http://code.google.com/p/javaparser/}{http://code.google.com/p/javaparser/}
    \item projekt na GoogleHosting
    \item v podstatě gramatika pro JJTree vytvářející strom tříd a objektů, který je možné procházet pomocí visitor patternu
    \end{itemize}
  \end{itemize}
\item použití prostředků platformy NetBeans
  \begin{itemize}
  \item Retouche API
  \item \href{http://bits.netbeans.org/6.9.1/javadoc/}{http://bits.netbeans.org/6.9.1/javadoc/}
  \end{itemize}
\item použití prostředků poskytovaných platformou Java 6 (Sun verze) \cite{source_code_analysis_corejavatechtips}
  \begin{itemize}
  \item \emph{JSR 199 -- Java Compiler API}
    \begin{itemize}
    \item volání překladače jazyka Java pomocí API ze zdrojového kódu programu
    \item balíček \verb+javax.tools+
    \end{itemize}
  \item \emph{JSR 269 -- Pluggable Annotation Processing API}
    \begin{itemize}
    \item možnost přidání vlastního kódu pro zpracovávání anotací/kódu do instance překladače
    \item balíček \verb+javax.annotation.processing+ -- zpracovávání anotací
    \item balíček \verb+javax.lang.model+ -- třídy poskytující model pro syntaktické elementy jazyka Java
    \end{itemize}
  \item \emph{Compiler Tree API}
    \begin{itemize}
    \item nestandardní rozšíření Java JDK
    \item \href{http://download.oracle.com/javase/6/docs/jdk/api/javac/tree/index.html}{http://download.oracle.com/javase/6/docs/jdk/api/javac/tree/index.html}
    \item balíček \verb+com.sun.source.tree+ -- poskytuje rozhraní pro reprezentaci zdrojového kódu jako AST
    \item balíček \verb+com.sun.source.util+ -- poskytuje rozhraní pro operace nad AST
    \end{itemize}
  \end{itemize}
\end{itemize}

\chapter{Návrh}

\section{Návrh způsobu specifikace pravidel}

Abychom mohli specifikovat pravidla, musíme nejprve definovat strukturu, nad kterou budou tato pravidla platit. V tomto textu formalizujeme model programu pomocí teorie grafů. Nad grafem je možné dále specifikovat pravidla, která musí vstupní projekt splňovat.

\subsection{Formalizace modelu programu pomocí grafu}
\label{design-graph_formalization}

Analyzovaný softwarový projekt abstrahujeme jako orientovaný multigraf rozšířený o zobrazení množiny uzlů do množiny typů a zobrazení hran do množiny jejich klasifikátorů (označení typu vztahu mezi uzly). Dále přidáme ke každému vrcholu zobrazení, které mu přiřadí jméno (řetězec). Získáme tak následující strukturu:

\begin{displaymath}
  G = \langle V, E, \rho, K, C, N, \mathit{Kind}, \mathit{Class}, \mathit{Name}\rangle
  \label{extended_multigraph}
\end{displaymath}
v níž platí:
\begin{itemize}
\item $V$ je množina elementů (v našem případě části kódu)
\item $E$ je množina hran (v našem případě vztahy mezi částmi kódu - např. volání funkce, dědičnost)
\item $V \cap E = \emptyset$
\item $\rho: E \mapsto V \times V$ je zobrazení množiny hran do množiny uspořádaných dvojic vrcholů (incidence)
\item $K$ je libovolná množina označení typů vrcholů\footnote{Pod pojmem typ zde rozumíme jakékoliv označení, které specifikuje o jaký objekt se jedná -- může to být třída, metoda, příkaz, \ldots},
\item $\mathit{Kind}: V \mapsto K$ je zobrazení, které přiřadí každému vrcholu jeho typ,
\item $C$ je množina klasifikátorů hran,
\item $N$ je množina jmen (řetězců)
\item $\mathit{Class}: E \mapsto C$ je zobrazení, které přiřadí každé hraně její klasifikátor (zda se jedná o \emph{method call}, \emph{dědičnost}, atd.)
\item $\mathit{Name}: V \mapsto N$ je zobrazení, které přiřadí vrcholu jeho jméno (např. jméno třídy, jméno metody, \ldots)
\end{itemize}

Ukázka formalizace zdrojového kódu pomocí grafu je na obrázku \ref{design-graph_example}. V uvedeném příkladě můžeme strukturu $G$ namapovat následujícím způsobem:
\begin{figure}[h!]
  \centering
  \includegraphics[width=1.0\textwidth]{./graphs/graph_example.png}
  \caption{Příklad formalizace hierarchie tříd jako grafu.\label{design-graph_example}}
\end{figure}
Množinu vrcholů můžeme ztotožnit s množinou názvů elementů (v našem případě třídy, pole, metody)\footnote{V konečném důsledku bude tato množina představována konkrétními elementy tak, jak se nalézají ve zdrojovém kódu.}.
\begin{align*}
  V = &\{ \\
  &Vehicle, Car, MotorCycle, Bicycle, MountainBike, RacingBike, \\
  &wheelNumber, doorNumber, \\
  &controlGear, gearUp, gearDown \\
  &\}
\end{align*}
Abychom mohli demonstrovat množinu hran, bylo provedeno očíslování. Hranu zde identifikujeme číslem. V počítačové reprezentaci se může jednat o konkrétní objekt uložený v poli. Množinu hran potom můžeme reprezentovat prostým výčtem čísel:
\begin{displaymath}
  E = \{1, 2, 3, 4, 5, 6, 7, 8, 9, 10, 11, 12\}
\end{displaymath}
Zobrazení $\rho$ můžeme formálně zapsat jako následující množinu\footnote{Zobrazení je binární relace. Proto jej můžeme reprezentovat jako množinu uspořádaných dvojic.}:
\begin{align*}
  \rho = &\{ \\
  &(1, (Bicycle, Vehicle)), \\
  &(2, (Car, Vehicle)), \\
  &(3, (MotorCycle, Vehicle)), \\
  &(4, (MountainBike, Bicycle)), \\
  &(5, (RacingBike, Bicycle)), \\
  &(6, (Vehicle, wheelNumber)), \\
  &(7, (Car, gearUp)), \\
  &(8, Car, controlGear)), \\
  &(9, (Car, doorNumber)), \\
  &(10, (Car, gearDown)), \\
  &(11, (controlGear, gearUp)), \\
  &(12, (controlGear, gearDown)) \\
  &\}
\end{align*}
Zbývají nám množina typů $K$,
\begin{align*}
  K = \{ class, public\_field, public\_method \}
\end{align*}
množina klasifikátorů hran $C$,
\begin{align*}
  C = \{ \langle\langle{}extends\rangle\rangle, \langle\langle{}has\_member\rangle\rangle, \langle\langle{}calls\rangle\rangle \}
\end{align*}
přiřazení typů uzlům $Kind$,
\begin{align*}
  Kind = &\{ \\
  &(Bicycle, class), \\
  &(Car, class), \\
  &(MotorCycle, class), \\
  &(MountainBike, class), \\
  &(RacingBike, class), \\
  &(Vehicle, class), \\
  &(wheelNumber, public\_field), \\
  &(doorNumber, public\_field), \\
  &(controlGear, public\_method), \\
  &(gearUp, public\_method), \\
  &(gearDown, public\_method) \\
  &\}
\end{align*}
a přiřazení klasifikátorů hranám:
\begin{align*}
  Class = &\{ \\
  &(1, \langle\langle{}extends\rangle\rangle), \\
  &(2, \langle\langle{}extends\rangle\rangle), \\
  &(3, \langle\langle{}extends\rangle\rangle), \\
  &(4, \langle\langle{}extends\rangle\rangle), \\
  &(5, \langle\langle{}extends\rangle\rangle), \\
  &(6, \langle\langle{}has\_member\rangle\rangle), \\
  &(7, \langle\langle{}has\_member\rangle\rangle), \\
  &(8, \langle\langle{}has\_member\rangle\rangle), \\
  &(9, \langle\langle{}has\_memeber\rangle\rangle), \\
  &(10, \langle\langle{}has\_member\rangle\rangle), \\
  &(11, \langle\langle{}calls\rangle\rangle), \\
  &(12, \langle\langle{}calls\rangle\rangle), \\
  \}
\end{align*}

%% Graf elementů použitých v kódu. Vrcholy $v \in V$ představují různé syntaktické elementy v analyzovaném kódu. Záleží na úrovni prováděné analýzy, jaké zvolíme vrcholy. V případě, že budeme analyzovat \uv{nízkoúrovňové} chování funkce, mohou být vrcholy jednotlivé příkazy a jejich části. Pokud budeme analyzovat míru provázanosti tříd nebo metod, stačí když použijeme jako vrcholy třídy a metody (případně parametry metod.)
%% $G = (V, E)$

\subsection{Formalizace pravidel}

Nyní, když máme definován model, můžeme popsat jazyk pravidel, která mají pro daný model platit. Vytvoříme jednoduchý jazyk (DSL) pro specifikaci validačních parametrů. Jazyk popíšeme jednoduchou gramatikou (listing \ref{design-grammar}). U této gramatiky předpokládáme, že bílé znaky jsou ignorovány (pokud nejsou uvnitř řetězce).

TODO: definovat základní elementy jazyka pravidel

\begin{itemize}
\item co jsou to selektory (napsat definici platnou pro tento dokument),
\item co jsou to predikáty (napsat definici platnou pro tento dokument),
\item popis základní struktury souboru s definicí validačních pravidel,
\item ukázka dokumentu odpovídajícího specifikované gramatice.
\end{itemize}

%% TODO: přepsat jako přesné definice, které se budou v dalším textu dodržovat
\subsubsection{Typy operátorů}
\begin{itemize}
\item Predikát (Predicate) -- operátor, který na základě vstupních parametrů a poskytované rozhodovací funkce vrací booleovskou hodnotu (true/false)
\item Vrcholový selektor (Vertex selector) -- vrací množinu vrcholů na základě vstupních parametrů a poskytované výběrové funkce
\item Hranový selector (Edge selector) -- vrací množinu hran na základě vstupních parametrů a poskytované výběrové funkce
\end{itemize}

%% TODO: na tomto místě popsat formát pravidel matematicky

Serializace pravidel ve formátu vhodném pro zadávání do počítače je přesně popsána pomocí gramatiky specifikované v příloze \ref{avd_grammar}.

\subsection{Typy pravidel}
\begin{itemize}
\item pravidla definovaná ve vytvořeném formalismu (DSL rules)
\item uživatelsky programovaná pravidla (custom code java code, asserts)
\end{itemize}

TODO: zapracovat:

Ne všechna pravidla lze popsat exaktně ve smyslu splněno/nesplněno. Mnohdy můžeme kvalitu návrhu posuzovat pouze kvantitativně (dobrý návrh, lepší, moc vazeb, málo vazeb, atd.). Pro některé vlastnosti návrhu (např. low coupling, high cohesion) může být vhodnější poskytnou statistický přístup pro vyhodnocování (např. použití vhodného klasifikátoru).

\section{Návrh architektury systému}
\label{design-architecture}

Při návrhu architektury systému postupujeme metodou shora dolů. Nejprve budeme systém uvažovat jako jeden velký celek s globální zodpovědností (funkcionalitou). Tento velký blok následně dekomponujeme na jednotlivé komponenty, kterým přidělíme jasně definované zodpovědnosti.

Pro výsledný systém budeme používat kódové jméno ArchVal (Architecture validator), zkracované často v~názvech modulů jako \verb+av+.

Globální pohled na systém byl zaveden již v~části \ref{requirements-rules_evaluation} na obrázku \ref{requirements-system_structure}. Toto znázornění je velmi obecné. Vidíme však, že systém má dva hlavní vstupy\footnote{Jako vstupy/výstupy neuvažujeme další běžné součásti systému jako např. načítání konfiguračních souborů nebo logování.} a jeden výstup. Vstupy a výstupy představují podstatnou část doménových dat nad nimiž budeme dále pracovat. V~části \ref{design-domain_objects} popíšeme jednotlivé datové objekty, které postupně \uv{protékají} celým systémem. Ty budou popsány a dekomponovány jako první, protože jejich popis je nezbytný pro definici ostatních komponent systému.

Ve druhé fázi budeme dekomponovat jádro systému. Zodpovědnost jádra představuje zodpovědnost celého systému (zodpovědností celého systému je \uv{provést validaci projektu na základě množiny pravidel}. Tuto zodpovědnost dále rozdrobíme jednak na základní systémové komponenty a dále na množinu rozšíření, která bude možné postupně přidávat. Definujeme proto veřejná rozhraní (SPI) \cite{spi}, která budou implementována poskytovateli služeb.

%% \begin{itemize}
%% \item compiler
%% \item generátor vnitřní struktury (modelu) -- modelem je v našem případě graf
%%   \begin{itemize}
%%   \item \emph{vstup:} množina cest k \verb+*.java+ souborům (kompilačním jednotkám), z nichž se skládá softwarový projekt
%%   \item \emph{výstup:} graf, reprezentující konkrétní analyzovanou doménu vhodný pro další zpracování (graf podle \ref{design-graph_formalization} \nameref{design-graph_formalization})
%%   \end{itemize}
%% \item možnosti generování grafu závislostí mezi třídami (vnitřní reprezentace):
%%   \begin{itemize}
%%   \item on demand -- když bude potřeba, vybuduje se kompletně celý graf a poté se zahodí
%%   \item on the fly -- při práci uživatele na kódu se bude vždy generovat znovu celý graf
%%   \item kombinovaný přístup -- graf se ve vhodných okamžicích celý vybuduje a poté budou probíhat jeho následné aktualizace na základě uživatelovy práce nad kódem
%%   \end{itemize}
%% \item analyzátor
%% \item rozhraní - integrace do platformy NetBeans
%%   \begin{itemize}
%%   \item rozhraní pro zadávání pravidel
%%   \item rozhraní pro ověřování platnosti pravidel (validaci)
%%   \end{itemize}
%% \end{itemize}

% TODO: highlevel design $\rightarrow$ jaké budou moduly a co budou dělat (zatím bez konkrétní použité technologie)
% TODO: odstavec o modulech systému, pro každou část/modul napsat samostatnou podsekci, která popíše, jaké třídy modul obsahuje a jaké jsou jejich zodpovědnosti (responsibilities)

Návrh vysokoúrovňové struktury je na obrázku \ref{design-modules}.

\begin{figure}[h!]
  \centering
  \includegraphics[width=0.5\textwidth]{./uml/archval_module_cmp.png}
  \caption{Rozhraní pro rozšíření systému.\label{design-modules}}
\end{figure}

Je potřeba definovat je dostatečně obecně -- interfaces (možnost oddědění vlastního rozšiřujícího rozhraní).

% TODO: move to right position
Systém bude realizován pomocí rozšíření. Jednotlivé typy rozšíření jsou znázorněny na obrázku \ref{design-system_extensions}
\begin{figure}[h!]
  \centering
  \includegraphics[width=0.8\textwidth]{./graphs/system_extensions.png}
  \caption{Body rozšíření systému.\label{design-system_extensions}}
\end{figure}

Na základě výše uvedeného seznamu definujeme pro každý typ rozšíření interface:

\begin{itemize}
\item GraphGeneratorIface
\item OperatorIface
\item AnalysisIface
\item OutputGeneratorIface
%% TODO: rozhodnout, jak se budou lišit interface output generátoru pro ValidationReport a AnalysisReport, definovat výstupní formáty
\end{itemize}

\subsection{Doménové objekty}
\label{design-domain_objects}
% TODO: popsat jednotlivé

\begin{itemize}
\item \emph{Graph} (class)
\item \emph{Vertex} (class)
\item \emph{Edge} (class)
\item \emph{GraphModel} (class) -- množina grafů různých typů
\item \emph{ValidationModel} (class) -- AST strom souboru pravidel s navázanými operátory; bude se přegenerovávat při úpravě souboru/textu v GUI, atd\ldots
\item \emph{ValidationReport} (class)
\item \emph{AnalysisReport} (class)
\end{itemize}

% TODO: move to appropriate position:
Třídy \emph{ValidationModel} a \emph{GraphModel} zde zavádíme zejména proto, aby bylo možné tyto objekty uložit do vhodné cache a znovu je použít. Příkladem může být případ, kdy chceme tentýž projekt (pro nějž jsme již jednou generovali graf) validovat pomocí jiné/upravené množiny pravidel nebo naopak pokud chceme použít jednu množinu pravidel pro validaci více projektů. Bylo by zbytečné opakovaně provádět náročné operace generování grafu nebo parsování souboru pravidel a navazování operátorů.

\subsection{Modul av-core}
TODO: popis validačního procesu: rozepsat
\begin{itemize}
\item systém načte moduly
\item systém zkontroluje základní předpoklady - existuje alespoň jeden modul pro generování grafu, alespoň jeden modul operátorů (nebo poskytovatel analyzerů ???) a alespoň jeden modul s poskytovatelem generátou výstupu
\item systém zparsuje soubor pravidel a vygeneruje AST
\item systém zkontroluje, zda pro AST pravidel existují potřebné providery grafů
\item systém zkontroluje, zda pro AST pravidel existují všechny operátory
\item systém provede nabindování operátoru na uzly AST pravidel
\item systém vygeneruje pro projekt všechny požadované grafy
\item systém vyhodnotí AST nad grafy pomocí operátorů, zjištěná porušení pravidel bude zapisovat do validačního reportu
\item systém projde všechny analyzátory a nechá je vygenerovat analyzační report
\item systém pomocí výstupních modulů vygeneruje reporty (na základě konfigurace ???)
\end{itemize}

Jádro systému bude poskytovat třídy a rozhraní popsané v tabulce \ref{design-archval_core_components}.

\begin{table}
  \caption{Tabulka komponent jádra systému. \label{design-archval_core_components}}
  \begin{center}
    \begin{tabular}{| l | l | p{8cm} | }
      \hline
      \textbf{Název} & \textbf{Typ} & \textbf{Zodpovědnost} \\
      \hline
      \hline
      Action & class & integrace s běhovou platformou, vstupní/aktivační bod systému \\ \hline
      ValidationTask & class & řízení procesu validace \\ \hline
      Validator & class & bude provádět validaci pravidel v grafu na základě vstupních pravidel \\ \hline
      Analyzer & class & bude provádět statistické analýzy (pro návrhové principy, které nelze popsat přesnými pravidly) \\ \hline
      ConfigurationManager & class & komponenta zprostředkující konfigurační uživatelské volby jádru programu \\ \hline
      GraphGeneratorManager & class & zinicializuje jednotlivé moduly pro generování grafů a použije je pro vygenerování grafů (pokud budou tyto grafy potřeba) \\ \hline
      ReportGeneratorManager & class & použije některé z dostupných poskytovatelů pro generování výstupu pro vygenerování reportu \\ \hline
      GraphGeneratorIface & interface & rozhraní, které musí implementovat poskytoval generátoru grafu \\ \hline
      OperatorIface & interface & rozhraní, které musí implementovat poskytovatel balíčku operátorů \\ \hline
      ReportGeneratorIface & interface & rozhraní, které musí implementovat poskystovatel modulu pro generování výstupních reportů \\ \hline
      AnalysisIface & interface & rozhraní, které musí implementovat poskytovatel analýzy nad některým z grafů \\ \hline
      GraphGeneratorRegister & class & třída poskytující přístup k informacím o existujících poskytovatelích generátorů grafu \\ \hline
      OperatorRegister & class & třída poskytující přístup k informacím o existujících poskytovatelích operátorů \\ \hline
      AnalysesRegister & class & třída poskytující přístup k informacím o existujících poskytovatelích komponent analýzy \\ \hline
      ReportGeneratorRegister & class & třída poskytující přístup k informacím o existujících poskytovatelích generátorů výstupních reportů \\ \hline
    \end{tabular}
  \end{center}

\end{table}

%% TODO: add to table of important components
%% TODO: move to part about spi
\begin{itemize}
\item \emph{PredicateInterface}
\item \emph{VertexSelectorInterface}
\item \emph{EdgeSelectorInterface}

\item \emph{GraphModelGenerator} -- updateGraphModel() -- přegeneruje graph model -- pouze doplní požadované chybějící grafy, regenerate() -- kompletně zruší všechny grafy a vyganeruje nové (všechny požadované)
\end{itemize}

Na obrázku \ref{design-archval_core} jsou zobrazeny komponenty modulu \emph{archval-core}.

TODO: připsat k rozhraní graph generator: Reprezentuje komponentu, která provede převod zdrojových kódů na graf (zobrazení, které vybraným elementům AST přiřadí množinu vrcholů $V$ grafu $G$).

\begin{figure}[h!]
  \centering
  \includegraphics[width=1.0\textwidth]{./uml/archval_core_cmp.png}
  \caption{Komponenty jádra systému ArchVal.\label{design-archval_core}}
\end{figure}

TODO: přesunout do sekce popisující rozhraní \emph{GraphGeneratorInterface}:

Komponenta typu \emph{GraphGenerator} je znázorněna na obrázku \ref{design-graph_generator_io}.

\begin{figure}[h!]
  \centering
  \includegraphics[width=0.45\textwidth]{./graphs/graph_generator_io_graph.png}
  \caption{Znázornění vstupů a výstupů komponenty typu \emph{GraphGenerator}.\label{design-graph_generator_io}}
\end{figure}

\subsubsection{Action}
Integrace do běhové platformy. Reprezentuje uživatelskou akci. Bude implementováno na základě zvolené implementační platformy (akce v GUI).

% TODO: rozhodnout, jestli se bude jmenovat tato komponenta Controller nebo ValidationTask
\subsubsection{Controller / ValidationTask}
Samostatné vlákno s řízením celého validačního procesu. Synchronized sekce. Vstupem bude vhodná specifikace umístění analyzovaného projektu.

TODO: vyladit následující interface
\begin{itemize}
\item setProject()
\item startValidationThread()
\item cancelValidationThread()
\item getReport()
\end{itemize}

\subsubsection{Validator}
Vstupem bude soubor pravidel

Komponenta \emph{Validator} je znázorněna na obrázku \ref{design-validator_io}.

\begin{figure}[h!]
  \centering
  \includegraphics[width=0.45\textwidth]{./graphs/validator_io_graph.png}
  \caption{Znázornění vstupů a výstupů komponenty \emph{Validator}.\label{design-validator_io}}
\end{figure}

\begin{itemize}
\item registerOperatorPackage()
\item performValidation() : ValidationReport
\end{itemize}

Výstupem bude ValidationReport (vnitřní struktura, která se bude posléze převádět na vhodnou vnější reprezentaci).

\subsubsection{Analyzer}
TODO: analyzer providers ??? - analyzer nejspíš dostane k dispozici graf (opět musí specifikovat nad jakým grafem pracuje) a výstupem bude AnalysisReport (TBD!)

SPI -- poskytovatel služby, lookup, avd
% TODO: navrhnout aplikaci tak, aby mela vice rozhrani (nejen GUI, ale i moznost integrace, moznost textoveho vstupu pravidel)

\subsubsection{ValidationModelGenerator}
TODO: add name of this component to the table of components
TODO: add to appropriate place in the document

Znázornění vstupů a výstpů komponenty \emph{ValidationModelGenerator} je na obrázku \ref{design-validation_model_generator_io}.

% TODO: zkontrolovat, že 'výše popsaných pravidel' odkazuje na nějaká předchozí pravidla dříve v práci
Vstupem je množina pravidel zapsaná ve vhodné serializaci výše popsaných matematických pravidel.

\begin{figure}[h!]
  \centering
  \includegraphics[width=0.5\textwidth]{./graphs/validation_model_generator_io_graph.png}
  \caption{Znázornění vstupů a výstupů komponenty \emph{ValidationModelGenerator}.\label{design-validation_model_generator_io}}
\end{figure}

Různé vstupy: File, Stream, String, \ldots

\section{Návrh rozhraní pro zadávání pravidel}
TODO: zapracovat: U každého pravidla musí být deklarace typu grafu, nad nímž má platit.
%% rozhraní pro zadávání pravidel bude dáno jazykem, který se pro definici pravidel bude používat - možná kompilátor DSL jazyka, výrazy, atd.
\begin{itemize}
\item rozhraní pro zadávání pravidel
\item formát zadávání pravidel (serializace formalismu tak, aby jej bylo možné textově nebo jinak zadávat)
\end{itemize}

\section{Návrh rozhraní pro ověřování platnosti pravidel (validaci)}
\begin{itemize}
\item výstupní rozhraní
\item validační události
\end{itemize}
% TODO: navrhnout formát výstupu

%% TODO:
% typy grafů: class hierarchy graph, call graph
% call graph: wikipedia (<<invokes>> vazby)
%
% http://en.wikipedia.org/wiki/Call_graph
%

%% TODO: dat na spravne misto
\section{Návrh technologií pro implementaci}
\begin{itemize}
\item NetBeans platform
\item Maven project
\item Java 1.6
\item vstupem bude java maven projekt
\item \ldots
  % TODO: vytvorit prehled technologii i s jejich popisy
\end{itemize}

\chapter{Implementace}
% TODO: napsat přehled, co bude v této kapitole

\section{Specifikace pravidel požadovaných zadáním práce}
Pro specifikaci Demeter law definujeme predikát, který nám umožní vybrat vhodné množiny vrcholů. Vybrané množiny potom ověřeíme pomocí predikátu, který učí, zda jsou tyto množiny v pořádku, či zda porušují LoD princip.

\begin{definition}
Mějme graf $G = \langle V, E, \rho, K, C, \mathit{Kind}, \mathit{Class}\rangle$ se zobrazeními definovanými dříve. Definujme selektor $F(G, v', k', c')$, $v' \in V$, $k' \in K$, $c' \in C$ jako množinu vrcholů grafu $G$, které jsou dostupné z vrcholu $v'$ pomocí orientované cesty, pro jejíž všechny vrcholy $v''$ s~výjimkou posledního platí $Kind(v) \ne k'$ a která obsahuje hranu $e$, pro niž platí $Class(e) = c' $.
\end{definition}

\subsection{Law of Demeter}
Pro provedení validace princip LoD využijeme graf podobný tomu na obrázku \ref{implementation-lod_graph}. Lze jej získat ze zdrojových kódů pogramu. Vazba $\langle\langle{}uses\rangle\rangle$ představuje všechna použití (přístup k~public a protected polím a volání metod) jiných tříd v rámci metody.

\begin{figure}[h!]
  \centering
  \includegraphics[width=1.0\textwidth]{./graphs/demeter_graph.png}
  \caption{Graf použitý pro validaci principu LoD.\label{implementation-lod_graph}}
\end{figure}

Pravidlo LoD potom specifikujeme nad grafem $G = \langle V, E, \rho, K, C, \mathit{Kind}, \mathit{Class}\rangle$ následovně:

\begin{align*}
\forall v \in V: Kind(v) = class\\
\end{align*}
\begin{align*}
[((&F(G, v, class, \langle\langle{}has\_field\rangle\rangle{}) \cup F(G, v, class, \langle\langle{}has\_param\rangle\rangle{}) \cup\\
&F(G, v, class, \langle\langle{}instantiates\rangle\rangle{})) \cap F(v, class, \langle\langle{}uses\rangle\rangle{}) \setminus \{v\}] = \emptyset
\end{align*}

Specifikované pravidlo vyjadřuje požadavek, aby množina vrcholů do níž se dostaneme pomocí hran, které představují povolené vstupy tříd do analyzované třídy (třídní proměnné, parametry, vytvářené objekty), byla totožná s množinou všech vrcholů, které představují všechny třídy používané (klasifikátor $\langle\langle{}uses\rangle\rangle$) v rámci některé z metod. Odečtení vrcholu $v$ z výsledné množiny je zde kvůli tomu, abychom nemuseli navíc zavádět zbytečnou hranu identifikující, že třída je známá sama sobě.

\subsection{Low coupling}
% TODO:

\subsection{High cohesion}
% TODO:

\section{Implementace jádra systému}
% TODO: describe technologies used to implement parser and analysis part
% TODO: moduly popsané v části \ref{design-architecture} implementujeme jako jednotlivé moduly platformy NetBeans.

\subsection{Implementace registrů poskytovatelů služeb}
\begin{itemize}
\item SPI
\item NetBeans lookup -- /META-INF/services
\end{itemize}

\subsection{Implementace základních operátorů}
Vyhodnocení univerzálního kvantifikátoru $\forall v \in V$ lze přepsat jako jednoduchý cyklus přes vrcholy vrácené zpřesňující operátorem, která vybírá nějakou podmnožinu z množiny vrcholů analyzovaného grafu. Získáme tak kód podobný listingu \ref{listing-forall}. V tomto kódu, který je fragmentem metody používáme symoblický název \verb+condition+ pro podmínku, kterou musí splňovat každý prvek \emph{v} iterované kolekce vrcholů \emph{vertices}.

\begin{lstlisting}[
    language=java,
    caption={Implementace univerzálního kvantifikátoru $\forall$.},
    label=listing-forall
  ]
for (Vertex v : vertices) {
    if (!condition(v, ...)) {
        return false;
    }
}
return true;
\end{lstlisting}

Analogicky budeme postupovat u existenčního kvantifikátoru $\exists$. Zde nám stačí nalézt alespoň jeden element, pro který vyhodnocovaná vlastnost platí. Listing \ref{listing-exists} představuje fragment metody. Pokud se podaří nalézt alespoň jeden prvek, který splňuje požadovanou podmínku, metoda vrátí hodnotu \emph{true}. Vzhledem k použití příkazu \emph{return} je zřejmé, že dochází k \uv{línemu} vyhodnocování -- vyhodnocení je ukončeno nalezením prvního vyhovujícího elementu (další se neprohledávají). Stejně jako u operátoru $\forall$ i zde název \verb+condition+ symbolizuje konkrétní podmínku, která má platit nad alespoň jedním prvkem množiny uzlů.

\begin{lstlisting}[
    language=java,
    caption={Implementace existenčního kvantifikátoru $\exists$.},
    label=listing-exists
  ]
for (Vertex v : vertices) {
    if (condition(v, ..)) {
        return true;
    }
}
return false;
\end{lstlisting}

Můžeme si všimnout, že se obě implementace liší pouze přehozením podmínek -- zatímco v prvním případě ($\forall$) musíme projít všechny prvky, abychom zjistili, zda všechny prvky splňují požadovanou vlastnost, ve druhém případě ($\exists$) budeme všechny prvky procházet pouze v krajním případě, kdy vlastnost neplatí pro žádný z prvků.

% TODO: přesunout do sekce týkající se generátorů grafu (možná dokonce do návrhu nebo analýzy)
% TODO: roztřídit
\begin{itemize}
\item Java 6 Compiler API (JSR 199) \cite{apidoc:java6api}
\item annotation processor (JSR 269)
\item Compiler Tree API
  \begin{itemize}
  \item \verb+com.sun.source.tree+
  \item \verb+com.sun.source.util+
  \end{itemize}
\item je nutné použít nestandardní API, které je součástí Sun Javy - balíčky \verb+com.sun.*+ (je potřeba přidat \$\{java\_home\}/../lib/tools.jar do classpath)
\end{itemize}

\section{Implementace modulů rozšíření}

\subsection{av-graphgen-demeter}
ukázkový poskytovatel generátoru grafu

\emph{DemeterGraphGeneratorProvider} implementuje \emph{GraphGeneratorIface}.

\subsection{av-operators-demeter}
ukázkový balíček operátorů pro ověřování principu LoD

Obsahuje sadu tříd implementující operátory potřebné pro definování pravidla pro validaci LoD. Tyto třídy implementují rozhraní \subsubsection{OperatorIface} (resp. jeho podtypy).

\subsection{av-analysis-lowcoup}
ukázková analýza low coupling

\subsection{av-analysis-highcoh}
ukázková analýza high cohesion

\section{Integrace do NetBeans IDE}

Při implementaci bylo hojně využíváno informací z \cite{netbeans_platform}.

% TODO: write about actions, about usage of the ArchVal API
Seznam integračních komponent je k dispozici v tabulce \ref{implementation-integration_components}.

\begin{table}
  \caption{Tabulka integračních komponent systému. \label{implementation-integration_components}}
  \begin{center}
    \begin{tabular}{ | l | l | p{8cm} | }
      \hline
      \textbf{Název} & \textbf{Typ} & \textbf{Zodpovědnost} \\
      \hline
      \hline
      Action & class & integrace s běhovou platformou, vstupní/aktivační bod \\ \hline
      %% TODO: 'třída implementující rozhraní...'
      GraphGeneratorRegister & class & třída poskytující přístup k informacím o existujících poskytovatelích generátorů grafu \\ \hline
      OperatorRegister & class & třída poskytující přístup k informacím o existujících poskytovatelích operátorů \\ \hline
      AnalysesRegister & class & třída poskytující přístup k informacím o existujících poskytovatelích komponent analýzy \\ \hline
      OutputGeneratorRegister & class & třída poskytující přístup k informacím o existujících poskytovatelích generátorů výstupních reportů \\ \hline
    \end{tabular}
  \end{center}
\end{table}

\subsection{Action}
Integrace do běhové platformy. Reprezentuje uživatelskou akci. Bude implementováno na základě zvolené implementační platformy (akce v GUI).

\subsection{Podpora editace AVD souborů}
\begin{itemize}
\item experimentální mime typ \verb+text/x-avd+
\item zvýrazňování syntaxe
\end{itemize}

\subsection{Výstupní rozhraní}
Podpora pro různé formáty výstupu. Popsat vnitřní reprezentaci reportu, tak aby jej bylo možné převést na vhodnou vnější reprezentaci. Podpora pro dopsání různých výstupních formátu jako např.:
\begin{itemize}
\item HTML
\item prostý text
\item zobrazení pomocí swing komponent (vygenerování modelu pro JTree)
\item \ldots
\end{itemize}

\subsection{Konfigurace}
\begin{itemize}
\item konfigurovatelné položky
\item konfigurační dialog
\item způsob ukládání konfigurace (podporováno platformou NB)
\end{itemize}

\chapter{Testování}

\begin{itemize}
 \item Způsob, průběh a výsledky testování.
 \item Srovnání s existujícími řešeními, pokud jsou známy.
\end{itemize}

%% TODO:
% - unit testing
% - validation of resulting software product
% - ...

\chapter{Závěr}

\begin{itemize}
\item Zhodnocení splnění cílů DP/BP a  vlastního přínosu práce (při formulaci je třeba vzít v potaz zadání práce).
\item Diskuse dalšího možného pokračování práce.
\end{itemize}


% TODO: complete references
%\bibliographystyle{abbrv}

\bibliographystyle{csplainnat}

%bibliographystyle{plain}
%\bibliographystyle{psc}
{
  %JZ: 11.12.2008 Kdo chce mit v techto ukazkovych odkazech take odkaz na CSTeX:
  \def\CS{$\cal C\kern-0.1667em\lower.5ex\hbox{$\cal S$}\kern-0.075em $}
  \bibliography{reference}
}

% M. Dušek radi:
%\bibliographystyle{alpha}
% kdy citace ma tvar [AutorRok] (napriklad [Cook97]). Sice to asi neni  podle ceske normy (BTW BibTeX stejne neodpovida ceske norme), ale je to nejprehlednejsi.
% 3.5.2009 JZ polemizuje: BibTeX neobvinujte, napiste a poskytnete nam styl (.bst) splnujici citacni normu CSN/ISO.

\appendix

\chapter{Seznam použitých zkratek}

\begin{description}
\item[API] Application Programming Interface
\item[AST] Abstract Syntax Tree
\item[AVD] ArchVal definition -- zkratka používaná pro označení souboru specifikace pravidel
\item[BNF] Backus-Naur form -- standardní forma zápisu bezkontextových gramatik
\item[CORBA] Common Object Request Broker Architecture
\item[DAO] Data Access Object
\item[DAO] Data Access Object
\item[DOM] Document object model
\item[DTD] Document Type Definition
\item[EBNF] Extedned Backus-Naur form -- novější způsob specifikace bezkontextových gramatik
\item[IDE] Integrated development environment
\item[JDK] Java Development Kit
\item[LoD] Law of Demeter
\item[PDF] Portable Document Format
\item[SPI] Service Provider Interface
\item[UML] Unified Modeling Language
\item[XML] Extensible Markup Language
\item[av] ArchVal -- kódové označení vyvíjeného nástroje
\end{description}

%\chapter{UML diagramy}
\textbf{\large Tato příloha není povinná a zřejmě se neobjeví v každé práci. Máte-li ale větší množství podobných diagramů popisujících systém, není nutné všechny umísťovat do hlavního textu, zvláště pokud by to snižovalo jeho čitelnost.}

\chapter{Instalační a uživatelská příručka}
\textbf{\large Tato příloha velmi žádoucí zejména u softwarových implementačních prací.}

\chapter{Gramatika jazyka pro specifikaci pravidel}
\label{avd_grammar}
\lstinputlisting[
  label=avdgrammar,
  caption={Gramatika jazyka pro specifikaci pravidel (AVD soubory).}
]{./listings/ArchvalRulesetGrammar.g}

\chapter{Nástroje použité pro realizaci diplomové práce}
TODO: write short text describing environment which was used for work realization

\begin{itemize} 
\item NetBeans platform, version
\item Linux, version
\item Git, version
\item latex, pdflatex, latexmk (tikz, ...)
\item GraphViz
\end{itemize}

\chapter{Obsah přiloženého CD}

Obsah přiloženého CD je popsán v~tabulce \ref{cd_files_table}.

\begin{table}[ht]
  \centering
  \begin{tabular}{|l|l|p{20em}|}
    \hline
    jméno & typ & popis \\
    \hline
    \hline
    index.html & soubor & index stránka CD (relativní odkazy na další dokumenty v~adresáři \verb+./html+) \\
    \hline
    html & adresář & obsahuje soubory dokumentace projektu (javadoc pro \emph{av-core}) \\
    \hline
    project & adresář & zdrojové kódy všech modulů projektu \emph{ArchVal} \\
    \hline
    text & adresář & obsahuje text diplomové práce ve formátu \emph{pdf} \\
    \hline
    dist & adresář & obsahuje distribuční verzi projektu \emph{ArchVal} \\
    \hline
    samples & adresář & obsahuje příklady, na nichž byl nástroj \emph{ArchVal} testován \\
    \hline
  \end{tabular}
  \caption{Seznam souborů na CD. \label{cd_files_table}}
\end{table}


\end{document}
