\chapter{Instalační a uživatelská příručka}

\section{Instalace programu}

K~dispozici jsou dvě varianty programu (v adresáři \verb+dist+ na přiloženém cd). Soubor \verb+archval.zip+ obsahuje spustitelnou variantu projektu. Pro platformu Linux je možné využít instalátor (soubor \verb+achval-linux.sh+).

V~případě instalace ze zip souboru stačí tento soubor rozbalit do vhodného umístění. Pro spuštění instalátoru může být nutná nastavit \emph{executable bit} pomocí příkazu:

\begin{verbatim}
chmod u+x archval-linux.sh
\end{verbatim}

U~obou typů instalací je nutné provést konfiguraci systému v~souboru \verb+etc/archval.conf+ v~cílovém adresáři. Důvodem je přímá závislost projektu na souborech \emph{Sun JDK API}. V~souboru je nutné nastavit proměnnou \verb+jdkhome+ na úplnou cestu k~JDK.

Program je možné spustit pomocí spouštěcích souborů, které jsou k dispozici v~adresáři \verb+bin/+ (verze pro Linux a pro Windows).

\section{Používání programu}
Po spuštění programu je k~dispozici základní verze NetBeans IDE. Systém podporuje jednak vytváření a editaci AVD souborů a dále provedení validace Maven Java projektů na základě existujícího AVD souboru.

Práce s~AVD soubory je podobná práci s~jinými typy souborů na platformě NetBeans. Nový AVD soubor vytvoříme pomocí posloupnosti akcí \emph{File | New file... | Other | Empty AVD file}.

Editace souborů je stejná jako pro kterýkoliv jiný soubor projektu -- stačí je otevřít buď poklepáním nebo pomocí pravého tlačítka myši. Soubor je otevřen v~editoru, který podporuje zvýrazňování AVD syntaxe.

Máme-li k~dispozici AVD soubor, můžeme jej použít k~validaci projektu, který je vybrán jako hlavní projekt. Klepneme pravým tlačítkem na AVD soubor a vybereme položku \emph{Validate main project}. Není-li vybrán žádný projekt jako hlavní, je zobrazena informace o~absenci výběru ve status baru a není provedena žádná akce. V~opačném případě je provedena validace projektu a výstup je zobrazen ve standardním výstupním okně NetBeans IDE.

\section{Gramatika jazyka pro specifikaci pravidel}
\label{avd_grammar}
Výpis \ref{avdgrammar} poskytuje k~dispozici gramatiku jazyka AVD pro specifikaci pravidel. Tuto gramatiku je možné používat jako referenci při psaní AVD specifikací a také při realizaci rozšíření.
\vspace{0.5cm}
\lstinputlisting[
  label=avdgrammar,
  caption={Gramatika jazyka pro specifikaci pravidel (AVD soubory).}
]{./listings/ArchvalRulesetGrammar.g}
