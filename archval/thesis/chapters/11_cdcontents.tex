\chapter{Obsah přiloženého CD}
\textbf{\large Tato příloha je povinná pro každou práci. Každá práce musí totiž obsahovat přiložené CD. Viz dále.}

Může vypadat například takto. Váš seznam samozřejmě bude odpovídat typu vaší práce. (viz \cite{infodp}):

\begin{figure}[h]
\begin{center}
\includegraphics[width=14cm]{figures/seznamcd}
\caption{Seznam přiloženého CD --- příklad}
\label{fig:seznamcd}
\end{center}
\end{figure}

Na GNU/Linuxu si strukturu přiloženého CD můžete snadno vyrobit příkazem:\\
\verb|$ tree . >tree.txt|\\
Ve vzniklém souboru pak stačí pouze doplnit komentáře.

Z \textbf{README.TXT} (případne index.html apod.)  musí být rovněž zřejmé, jak programy instalovat, spouštět a jaké požadavky mají tyto programy na hardware.

Adresář \textbf{text}  musí obsahovat soubor s vlastním textem práce v PDF nebo PS formátu, který bude později použit pro prezentaci diplomové práce na WWW.
