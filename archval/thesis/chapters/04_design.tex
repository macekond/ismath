\chapter{Návrh}

\section{Návrh způsobu specifikace pravidel}

\subsection{Formalizace pravidel}

\begin{itemize}
\item definice jazyka pro specifikaci pravidel
\item specifikovat množinu objektů, nad nimiž budou pravidla stavěna -- nad čím budeme operovat (možnosti: gramatika, třídy, FJ, graf, \ldots)
\item lze vyjít z Java Language Specification (množina neterminálních a terminálních symbolů)
\item může být nějaký DSL
\end{itemize}

\subsection{Typy pravidel}
\begin{itemize}
\item pravidla definovaná ve vytvořeném formalismu (DSL rules)
\item uživatelsky programovaná pravidla (custom code java code, asserts)
\end{itemize}

\section{Návrh architektury systému}
TODO: highlevel design $\rightarrow$ jaké budou moduly a co budou dělat (zatím bez konkrétní použité technologie)
\begin{itemize}
\item compiler
\item generátor vnitřní sturktury (modelu)
\item analyzátor
\item rozhraní
  \begin{itemize}
  \item rozhraní pro zadávání pravidel
  \item rozhraní pro ověřování platnosti pravidel (validaci)
  \end{itemize}
\end{itemize}

\section{Návrh rozhraní pro zadávání pravidel}
%% rozhraní pro zadávání pravidel bude dáno jazykem, který se pro definici pravidel bude používat - možná kompilátor DSL jazyka, výrazy, atd.
\begin{itemize}
\item rozhraní pro zadávání pravidel
\item formát zadávání pravidel (serializace formalismu tak, aby jej bylo možné textově nebo jinak zadávat)
\end{itemize}

\section{Návrh rozhraní pro ověřování platnosti pravidel (validaci)}
\begin{itemize}
\item výstupní rozhraní
\item validační události
\end{itemize}
