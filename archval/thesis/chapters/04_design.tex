\chapter{Návrh}

\section{Návrh způsobu specifikace pravidel}

\subsection{Formalizace pravidel}

TODO: vyjit z gramatiky, vytvorit nejakou mnozinu objektu nad nimiz budeme pracovat, vzit v uvahu FJ

\subsection{Typy pravidel}
\begin{itemize}
\item pravidla definovaná ve vytvořeném formalismu (DSL rules)
\item uživatelsky programovaná pravidla (custom code java code, asserts)
\end{itemize}

\subsection{Návrh rozhraní pro zadávání pravidel}
\begin{itemize}
\item rozhraní pro zadávání pravidel
\item formát zadávání pravidel (serializace formalismu tak, aby jej bylo možné textově nebo jinak zadávat)
\end{itemize}

\subsection{Návrh rozhraní pro ověřování platnosti pravidel (validaci)}
\begin{itemize}
\item výstupní rozhraní
\item validační události
\end{itemize}

%% TODO: rozsekat do jednotlivych sekci a podsekci a zpracovat
%% Součásti práce (samostatně realizovatelné celky):
%% - definovat jazyk pro definici pravidel
%%    - nad čím bude operovat (gramatika?, třídy?, FJ?)
%%    - lze vyjít z Java Language Specification (množina neterminálních a
%%      terminálních symbolů)
%%    - může být dsl
%% - rozhraní pro definici pravidel (bude dáno jazykem, který se pro
%%   definici pravidel bude používat - možná kompilátor DSL jazyka,
%%   výrazy, atd.)
%% - rozhraní pro validaci
