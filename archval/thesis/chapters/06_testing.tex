\chapter{Testování}

Výsledný systém byl testován jednak během vývoje a následně i po dokončení. Testování zahrnovalo jednak \emph{testy jednotek}, které jsou popsány v~první sekci této kapitoly a dále \emph{ověřování činnosti nástroje na vzorových příkladech kódu} jak bylo požadováno zadáním práce.

Pro ověření činnosti nástroje byly vytvořeny příklady různých typů. Zabýváme se zde příklady, které princip LoD porušují, dále příklady, pro něž systém vyhodnocuje chybně správné příklady jako nevalidní a speciálními případy.

\section{Testování jednotek}
Pro zajištění kvalitního zdrojového kódu byly použity testy jednotek. Fáze návrhu již bere jednotkové testování v~potaz. Díky specifikaci rozhraní pro většinu komponent bylo možné poskytnout vlastní testovací implementace objektů (mock objekty). Například třída \emph{MockOperatorsRegister} implementuje rozhraní \emph{OperatorsRegisterIface} a poskytuje falešné instance operátorů, aby bylo možné otestovat, zda vytváření vyhodnocovacího stromu probíhá dle očekávání.

Vzhledem k~časové náročnosti realizace testů jsou napsány pouze testy klíčových komponent systému. Jednotkové testování bylo velmi pohodlným nástrojem zejména v~počátečních fázích projektu, kdy celý projekt nebyl spustitelný a nebylo jej možné otestovat navíc ručně.

Kromě testování jednotek byla provedena rešerše možností testování na platformě NetBeans. Platforma NetBeans v~posledních verzích podporuje nejen jednotkové testování ale i testy funkčnosti prostřednictvím tzv. operátorů. Tyto operátory jsou schopné realizovat uživatelské akce jako výběr položky z~menu, založení projektu atd. Byl realizován pouze elementární test funkčnosti, který sloužil ke kontrole toho, jestli se aplikace správně spustí (podaří se načtení všech modulů).

\section{Ověřování funkčnosti na příkladech}
Abychom demonstrovali funkčnost vytvořeného systému, bylo provedeno testování na konkrétních příkladech. Tyto příklady jsou zahrnuty v~podadresáři \verb+samples+ na CD, které je nedílnou součástí této práce. Podívejme se na jednotlivé případy, které byly ověřovány:

\subsection{Příklady porušení principu LoD}

Ukázka \verb+method_return_access+ demonstruje přístup k~rozhraní třetí třídy na základě reference vrácené voláním metody na jiné třídě, k~níž máme přístup. Podobně funguje i ukázka \verb+member_field_access+.

Příklad \verb+hidden_violation+ ukazuje další porušení principu LoD. Současně je zde demonstrováno, jak by mělo zpracování probíhat správně.

Všechny tyto příklady vyhodnotí ArchVal správně jako \verb+Violation found+.

\subsection{Falešně pozitivní případy}

Na základě testování se ukázalo, že systém pro některé příklady, které princip LoD neporušují vrací informaci o~porušení tohoto principu. Při bližším zkoumání se ukázalo, že se jedná o~výjimky, jejichž použití je běžné nebo zvláštní vlastnosti jazyka, s~nímž nepočítá generátor grafu.

Příklad \verb+string_literal_invalid+ demonstruje falešně pozitivní detekci porušení principu LoD ačkoliv z~hlediska LoD se jedná o~případ, kdy konstruujeme nový objekt. Protože ale generátor grafu vyhledává volání konstruktorů, je konstrukce řetězce pomocí uvozovek vynechána. Pokud bychom nahradili tyto řetězce voláním konstruktoru pro objekt \verb+String+, systém bude reagovat správně.

Dalším příkladem, který vyhodnotí systém jako porušení je běžný výpis řetězce na standardní výstup:

\begin{verbatim}
System.out.println()
\end{verbatim}

Pro nasazení na reálných projektech je nutné tyto výjimky zohlednit a uměle rozšířit množiny hran v~grafu tak, aby tyto případy byly ignorovány.

\subsection{Speciální případy}

Kromě testování na běžných projektech bylo provedeno ještě testování na specifickém případu, kdy projekt obsahuje prázdné \verb+*.java+ soubory nebo soubory, které obsahují pouze deklarace \verb+import+. Systém se v~takových případech choval správně -- žádné porušení principu LoD nebylo detekováno.
