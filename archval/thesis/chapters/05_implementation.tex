\chapter{Implementace}

TODO: Popis implementace/realizace se zaměřením na nestandardní části řešení.

\section{Specifikace pravidel požadovaných zadáním práce}
Pro specifikaci Demeter law definujeme predikát, který nám umožní vybrat vhodné množiny vrcholů. Vybrané množiny potom ověřeíme pomocí predikátu, který učí, zda jsou tyto množiny v pořádku, či zda porušují LoD princip.

\begin{definition}
Mějme graf $G = \langle V, E, \rho, K, C, \mathit{Kind}, \mathit{Class}\rangle$ se zobrazeními definovanými dříve. Definujme selektor $F(G, v', k', c')$, $v' \in V$, $k' \in K$, $c' \in C$ jako množinu vrcholů grafu $G$, které jsou dostupné z vrcholu $v'$ pomocí orientované cesty, pro jejíž všechny vrcholy $v''$ s~výjimou posledního platí $Kind(v) \ne k'$ a která obsahuje hranu $e$, pro niž platí $Class(e) = c' $.
\end{definition}

\subsection{Law of Demeter}
Pro provedení validace princip LoD využijeme graf podobný tomu na obrázku \ref{implementation-lod_graph}. Lze jej získat ze zdrojových kódů pogramu. Vazba $\langle\langle{}uses\rangle\rangle$ představuje všechna použití (přístup k~public a protected polím a volání metod) jiných tříd v rámci metody.

\begin{figure}[h!]
  \centering
  \includegraphics[width=1.0\textwidth]{./graphs/demeter_graph.png}
  \caption{Graf použitý pro validaci principu LoD.\label{implementation-lod_graph}}
\end{figure}

Pravidlo LoD potom specifikujeme nad grafem $G = \langle V, E, \rho, K, C, \mathit{Kind}, \mathit{Class}\rangle$ následovně:

\begin{align*}
\forall v \in V: Kind(v) = class\\
\end{align*}
\begin{align*}
[((&F(G, v, class, \langle\langle{}has\_field\rangle\rangle{}) \cup F(G, v, class, \langle\langle{}has\_param\rangle\rangle{}) \cup\\
&F(G, v, class, \langle\langle{}instantiates\rangle\rangle{})) \cap F(v, class, \langle\langle{}uses\rangle\rangle{}) \setminus v] = \emptyset
\end{align*}

Specifikované pravidlo vyjadřuje požadavek, aby množina vrcholů do níž se dostaneme pomocí hran, které představují povolené vstupy tříd do analyzované třídy (třídní proměnné, parametry, vytvářené objekty), byla totožná s množinou všech vrcholů, které představují všechny třídy používané (klasifikátor $\langle\langle{}uses\rangle\rangle$) v rámci některé z metod. Odečtení vrcholu $v$ z výsledné množiny je zde kvůli tomu, abychom nemuseli navíc zavádět zbytečnou hranu identifikující, že třída je známá sama sobě.

\subsection{Low coupling}
TODO:

\subsection{High cohesion}
TODO:

\section{Implementace jádra systému}
TODO: describe technologies used to implement parser and analysis part

TODO: moduly popsané v části \ref{design-architecture} implementujeme jako jednotlivé moduly platformy NetBeans.

Integrace s platformout NetBeans - třída Action.

\begin{itemize}
\item Java 6 Compiler API (JSR 199) \emph{\{TODO: add reference\}}
\item annotation processor (JSR 269) \emph{\{TODO: add reference\}}
\item Compiler Tree API
  \begin{itemize}
  \item \verb+com.sun.source.tree+
  \item \verb+com.sun.source.util+
  \end{itemize}
\end{itemize}

TODO: zapracovat:

je nutné použít nestandardní API, které je součástí Sun Javy - balíčky \verb+com.sun.*+ (je potřeba přidat \$\{java\_home\}/../lib/tools.jar do classpath)

lze se inspirovat v programu Sorcerer (\href{https://sorcerer.dev.java.net/}{https://sorcerer.dev.java.net/}), který pomocí Java Tree API provádí zpracovávání Java kódu a generování přesné HTML dokumentace. \emph{TODO: in case of usage of this material, add reference!}
