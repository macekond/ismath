\chapter{Analýza}
%% TODO: zde napsat prehledovy text o tom, co vsechno tato kapitola obsahuje

%% Analýza a návrh implementace (včetně diskuse různých alternativ a volby implementačního prostředí).

%% V části analýzy provést rozbor jednotlivých principů + výsledky rešerší - využívat hojně informace z článků a důsledně citovat zdroje

\section{Analýza principů objektového návrhu}

%% PRINCIPY versus PATTERNY -> citovat clanek

\subsection{Analyzované principy}
Ukázkové návrhové principy analyzované v rámci této práce jsou znázorněny na obrázku \ref{analyzed_principles}. Poznamenejme, že tzv. Demeterův zákon je speciálním případem pravidla pro \uv{low coupling}.

\begin{figure}[h!]
\centering
\includegraphics[width=0.5\textwidth]{./graphs/oop_design_principles.png}
\caption{Znázornění analyzovaných návrhových principů.\label{analyzed_principles}}
\end{figure}

% TODO: v rámci každého návrhového principu uvést příklad porušení
% tohoto principu (případně i příklad, který tento princip dodržuje)

% Příklad pravidla:
% ``Třídy z balíčku A nesmí záviset na jiných konkrétních třídách, ale nejvýše na rozhraních balícků B.''
%  (programování proti rozhraní namísto proti kokrétní implementaci)"

\subsubsection{Low coupling (nízká závislost/vazba)}
Důležitou návrhovou zásadou je snaha snížit provázanost modulů na minimum. Je možné kategorizovat způsob provázanosti modulů do různých skupin. Následující přehled je převzat z \cite{wiki:coupling} (ještě jemnější dělení je uváděno v \cite{STVR:STVR162}).

\begin{itemize}
\item\emph{Content coupling (nejvyšší forma závislosti)} -- závislost na obsahu modulu -- jeden modul modifikuje nebo se spoléhá na vnitřní fungování jiného modulu (např. přístup k lokálním datům jiného modulu). V důsledku platí, že změní-li se způsob, kterým tento druhý modul produkuje data (umístění, typ, časování), povede to zcela jiste ke změnám v závislém modulu.
\item\emph{Common coupling} -- dva moduly sdílí stejná globální data (např. globální proměnnou), změna sdíleného globálního zdroje implikuje změny všech modulů, které je používají.
\item\emph{External coupling} -- dva moduly sdílí externě definovaný (standardizovaný) datový formát, komunikační protokol nebo rozraní zařízení.
\item\emph{Control coupling} -- jeden modul kontroluje tok druhého tím, že mu posílá informaci o tom, co má konat (např. předání \uv{to-do} příznaku).
% CLARIFY
\item\emph{Stamp coupling (Data-structured coupling)} -- mdouly sdílí složenou datovou strukturu a používají pouze její (často odlišnou) část (např. předávání kompletního záznamu funkci, která z něj potřebuje pouze jedno pole). Tato vazba může vést ke změně způsobu, kterým modul čte záznam, protože pole, které tento modul nepotřebuje bylo modifikováno.
% Stamp coupling is when modules share a composite data structure and use only a part of it, possibly a different part (e.g., passing a whole record to a function that only needs one field of it). This may lead to changing the way a module reads a record because a field that the module doesn't need has been modified.
\item\emph{Data coupling} -- moduly sdílí data pomocí parametrů. Každý parametr je elementární datový typ a jedná se o jediná data, která jsou sdílená (např. předávání celočíselné hodnoty funkci, která spočítá jeho druhou mocninu).
\item\emph{Message coupling (nejnižší forma závislosti)} -- provázanost modulů pouze pomocí zpráv, jedná se o nejnižší úroveň závislosti. Lze jí dosáhnout pomocí decentralizace stavu (u objektů), kde je komunikace dosahováno pomocí parametrů nebo předávání zpráv (\uv{message passing}).
\item\emph{No coupling} -- žádná závislost -- moduly spolu vůbec nekomunikují.
\end{itemize}

\subsubsection{High cohesion}

\subsubsection{Law of Demeter}

Existuje několik forem Demeterova zákona \cite{35588}, které jsou vhodné pro různé oblasti aplikace. Tyto typy jsou znázorněny na obrázku \ref{demeter_law_types}.

\begin{figure}[h!]
\centering
\includegraphics[width=0.7\textwidth]{./graphs/demeter_law_types.png}
\caption{Formy Demeterova zákona.\label{demeter_law_types}}
\end{figure}

Pro statickou analýzu lze použít \uv{class} formu Demeterova zákona.

\subsection{Ukázky kódu porušujících některá z pravidel}

\subsubsection{Porušení principu law of Demeter}

%% TODO: use some highlighter
%% TODO: add example of demeter law violation from paper notes
\begin{verbatim}
package handlers;

public class ...

\end{verbatim}

\section{Analýza problematiky v jazyce Java}

\subsection{Statický model programu v Javě}
% TODO: provide better name for this section
\subsubsection{Struktura Java projektu}

\subsubsection{Základní syntaktické elementy programovacího jazyka Java}
Grafické znázornění základních syntaktických elementů, jejichž struktura a názvy jsou převzaty z \cite{Gosling:2005:JLS:1036643}, je na obrázku \ref{toplevel_elements}.
% TODO: write some better accompanying text
\begin{figure}[h!]
\centering
\includegraphics[width=\textwidth]{./graphs/java_top_elements.png}
\caption{Struktura základních syntaktických elementů programovacího jazyka Java.\label{toplevel_elements}}
\end{figure}

Pro analýzu založenou na vyhledávání závislostí mezi třídami pro nás bude nejdůležitější syntaktický element \emph{TypeDeclaration}. Tento neterminální symbol se dále přepisuje na symboly uvedené na obrázku \ref{type_declaration_options}.

\begin{figure}[h!]
\centering
\includegraphics[width=\textwidth]{./graphs/toplevel_types.png}
\caption{Rozklad elementu TypeDeclaration.\label{type_declaration_options}}
\end{figure}

\section{Existující nástroje pro statickou analýzu kódu v Javě}
% TODO: add some background research on existing tools (take information from wiki, ../materials)
