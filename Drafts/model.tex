\documentclass[10pt,a4paper]{article}
\usepackage[ascii]{inputenc}
\usepackage{amsmath}
\usepackage{amsthm}
\usepackage{amsfonts}
\usepackage{amssymb}

\newtheorem{mydef}{Definition}

\begin{document}

\section{Data centric use case model}

\subsection{Prerequisites} 
\begin{enumerate}
	\item Set of string labels: $L$ 
	\item Set of basic types : $T$ 
	
\end{enumerate}
\subsubsection{Data model}
The data model is a platform independent model (PIM) of domain data.
$D = ( C$,$ A$,$ R$, $name$, $type$, $aclass$, $relation$, $card)$
\begin{itemize}
	\item $C$ is a set of classes,
	\item $A$ is a set of attributes,
	\item $R$ is a set of relation,
	\item $name$ is a function assigning names to classes, atributes and relations $name: C \cup A \cup R \rightarrow L$,
	\item $type$ assigns type to an attribute $type: A \rightarrow T$,
	\item $aclass: A \rightarrow C$ assigns atributes to a class,
	\item $relation: R \rightarrow C \times C$ connect two classes with a relation. The order is not importatnt in this relation: $r \in R, relation(r)=(C_1, C_2)=(C_2, C_1)$,
	\item $card: C \times R \rightarrow \langle N_0 \times (N \cup \lbrace * \rbrace) \rangle$ assigns cardinality to a relation.
\end{itemize}

\subsection{Interaction model}
The process model represents the interaction of the user and the system.
$I=($S$, $U$, $A$, $F$, $name$, $type$, $astep$, $flow$, $perform$)$
\begin{itemize}
	\item $S$ is a set of interaction steps. 
	\item $U$ is a set of users (or actors). The system is held as one of actors. The models with two actors are prefered because of their simplicity. 
	\item $A$ is a set of attributes,
	\item $F$ is a set of interaction flows,
	\item $name$ is a function assigning names to classes, atributes and relations $name: S \cup U \cup A \cup F \rightarrow L$,
	\item $type$ assigns type to an attribute $type: A \rightarrow T$,
	\item $astep: A \rightarrow S$ assigns attributes to a class,
	\item $flow: F \rightarrow S \times S$ connect two interaction steps in a flow. The order is importatnt in this relation: $f \in F, flow(f)=(S_1, S_2)\neq(S_2, S_1)$,
	\item $perform: S \rightarrow U$ assigns an user to a interaction step 
\end{itemize}

\begin{mydef}
Let call an \textbf{Interaction} each tuple $s_1, s_2 \in S: \exists f \in F: flow(f)=(s_1,s_2) $ and  $perform(s_1) \neq perform(s_2)$ 
\end{mydef}



\subsection{Use Case Model}

$M = (D'$, $I'$, $input$, $output)$
\begin{itemize}
	\item $D' \supset D $ is a data model
	\item $I' \supset I$ is a interaction model
	\item $input: D' \rightarrow S$ is a function assigning an input data to a interaction step
	\item $output: D' \rightarrow S$ is a function assigning an output data to a ineteraction step
\end{itemize}

\begin{mydef}
Atomic (or bidirectional) use case is an use case which interactions are only between two users.
\end{mydef}


\begin{mydef}
Composite UC is an use case which is compound from two or more atomic use cases.
\end{mydef}


\end{document}